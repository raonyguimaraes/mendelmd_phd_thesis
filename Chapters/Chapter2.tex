\chapter[Objetivos]{Objetivos}

\section{Objetivo Geral}

Estudar e desenvolver métodos e ferramentas que permitam o armazenamento e a análise de dados genômicos humanos possibilitando a investigação e o diagnóstico de casos clínicos de pacientes estudados pelo laboratório de uma maneira rápida e eficiente por médicos, cientistas e outros profissionais da área da Saúde. A seguir, apresentamos os objetivos específicos que foram definidos para serem realizados durante o desenvolvimento deste trabalho.

\section{Objetivos Específicos}

\begin{itemize}
  \item Desenvolver uma ferramenta que permita a médicos e cientistas investigarem os dados genômicos de seus pacientes.
  \item Gerar métricas de qualidade sobre o sequenciamento, a genotipagem e a cobertura do sequenciamento para cada indivíduo inserido no sistema.
  \item Desenvolver um \textit{pipeline} de anotação de variantes utilizando diferentes métodos e programas que ajudem na identificação de mutações que possam ser responsáveis por causar uma doença Mendeliana.
  \item Permitir a identificação de mutações em genes e doenças que já tiverem sido descritas pela literatura e de novas mutações e genes candidatos possivelmente associados com síndromes Mendelianas ainda não descritas pela literatura.
  \item Validar o programa desenvolvido com dados reais de pacientes do Laboratório de Genômica Clínica da Faculdade de Medicina da UFMG e com dados de exomas clínicos descritos anteriormente pela literatura.
  \item Estudar algoritmos de priorização de variantes como SIFT, Polyphen-2, CADD e Mutation Taster para estimar formas mais racionais de utilizar seus parâmetros para melhorar a análise dos dados.
\end{itemize}
