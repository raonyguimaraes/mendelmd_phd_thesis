\chapter{Conclusão e Perspectivas}

O Mendel,MD foi um exemplo prático de como é possível construir um software de que seja responsável pela aquisição, organização e análise de dados genômicos de diferentes casos clínicos que foram investigados pelo nosso laboratório. A disponibilização do código fonte de forma \textit{open-source} e pública na internet promete ajudar outros laboratório pelo mundo a analisar seus próprios casos clínicos e facilitar bastante a tarefa de identificar mutações candidatas para serem validadas experimentalmente.

O aumento da quantidade de dados biológicos disponíveis atualmente para serem analisados trouxe um grande desafio para a ciência atual. Vivemos em uma época onde os sequenciadores produzem muito mais dados do que os cientistas são capazes de analisar. Por causa disso é preciso que haja o desenvolvimento de novos algoritmos e tecnologias cada vez mais poderosas que sejam capazes de integrar diferentes tipos de experimentos (Ex. DNA, RNA e Proteína) para tentar extrair algum tipo de informação útil desses dados que possam ajudar diretamente no tratamento de pacientes. Isso promete causar uma revolução na Medicina nos próximos anos.

O uso do sequenciamento de exomas trouxe uma nova metodologia para a prática clínica que promete ajudar a solucionar casos clínicos que até hoje não foram solucionados por outros métodos convencionais. Além disso, o chamado \textit{screening} de crianças recém-nascidas, promete no futuro ajudar até mesmo no tratamento de algumas doenças genéticas e em alguns raros casos ajudar na definição do tipo de medicamento que pode ser utilizado para tratar cada paciente.

Atualmente, alguns pesquisadores sugerem que o exoma seja utilizado de uma maneira chamada de ``\textit{exome-first approach}''. Isso significa utilizar o sequenciamento de exomas antes mesmo de pedir outros exames clínico mais caros, com o objetivo de tentar identificar possíveis SNPs, indels ou CNVs que possam estar associadas de alguma forma com a doença do paciente. Quando isso acontecer, os médicos terão acesso não somente ao exoma do paciente mas também a todos os outros exomas que estarão presentes nos bancos de dados.

Com o aumento do número de indivíduos completamente sequenciados nos bancos de dados públicos, espera-se que o número de novas variantes encontradas em um único indivíduo diminua cada vez mais e com isso a tarefa de identificar variantes relevantes tende a ficar cada vez mais simples, especialmente para as variantes que sejam patogênicas, ou seja, aquelas variantes que já estejam associadas com doenças mendelianas pela literatura no OMIM ou no HGMD. Isso promete trazer grandes benefícios para o diagnóstico dos pacientes ao longo dos próximos anos.

A visualização dessa enorme quantidade de dados, também é uma área em constante evolução, e que promete facilitar cada vez mais a identificação de variantes clinicamente relevantes a partir dos dados genômicos de cada paciente. No futuro, ao consultarmos um médico será possível realizarmos perguntas mais complexas sobre a nossa saúde e com certeza a informação genômica poderá ser utilizada por diferentes especialistas para direcionar o tratamento de cada paciente de uma maneira um pouco mais individualizada.

Para concluir, o aumento do número de casos clínicos solucionados através do uso de sequenciamento de exomas promete continuar crescendo. Isso irá ajudar ainda mais a impulsionar a adoção deste método na prática clínica auxiliando o médico cada vez mais na hora de realizar o diagnóstico clínico de seus pacientes. Para que isso aconteça ainda será que preciso que esta técnica torne-se mais precisa e confiável e que novos programas sejam desenvolvidos para facilitar o acesso a este tipo de informação.