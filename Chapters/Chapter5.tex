\chapter{Discussão}

Os resultados apresentados no capítulo anterior fornecem evidências de que em alguns casos foi realmente possível realizar o diagnóstico clínico de pacientes com doenças Mendelianas através do uso do sequenciamento e análise de exomas. Ou seja, em alguns casos, mesmo com apenas um único indivíduo, foi possível identificar a mutação e o gene causador da doença investigada, sem que para isso fosse necessário utilizar mais de um indivíduo afetado para realizarmos a análise.

O desenvolvimento do Mendel,MD mostrou que é possível oferecer uma solução rápida e eficiente para a análise de exomas e ao mesmo tempo oferecer uma interface bastante simples e amigável aos seus usuários, de modo a permitir o acesso aos médicos e pesquisadores a essa enorme quantidade de dados que são gerados pelos sequenciadores de nova geração e também para incorporar a esses dados novas informações sobre a anotação dessas variantes que estejam presentes em diferentes tipos de bancos de dados.

Apesar da aplicação deste método ser relativamente nova na prática clínica, isso tem trazido enormes benefícios para auxiliar no diagnóstico clínico, e na solução de alguns casos que ainda não haviam sido resolvidos através de outros exames convencionais. É importante ressaltar que mesmo com a popularização do uso do sequenciamento de exomas, sempre será necessário confirmar a existência da mutação através de outros métodos de sequenciamento tradicionais, como por exemplo, utilizando o método de Sanger para fazer a validação e a confirmação da existência das mutações identificadas. Sem isso não seria possível chegar a um diagnóstico definitivo sobre o caso clínico estudado. Nos últimos dois anos, com o aumento da cobertura média dos exomas que recebemos para serem analisados (em média 100X), nós obtivemos uma melhora considerável o que facilitou ainda mais a análise e permitiu inclusive a identificação de algumas indels que possuíam uma boa cobertura no exoma.

A análise de exomas pode ser utilizada tanto para direcionar os estudos sobre uma doença específica, como por exemplo, para a identificação de um novo gene candidato que ainda não estiver descrito pela literatura científica como sendo associado com o fenótipo em estudo, ou então, para diagnosticar um paciente buscando por mutações que já estejam descritas pela literatura como sendo patogênicas e que já estejam presentes em bancos de dados públicos como o OMIM, HGMD ou Clinvar. Essa é uma parte essencial da e muito importante da ferramenta que foi desenvolvida.

\section{Sobre o armazenamento dos dados}

Em relação ao espaço para armazenamento dos dados genômicos o que muitas pessoas não percebem é que o tamanho dos arquivos para se armazenar as informações sobre cada indivíduo é considerado bastante pequeno. Sabemos que um genoma humano possui em média 4 milhões de variantes em relação ao genoma de referência que possui 4 bilhões de pares de base. Um exoma possui algo entre 20 e 40 mil variantes. Para armazenarmos a informação sobre um genoma humano em um arquivo VCF nós precisamos de apenas 40mb de espaço em disco. Como exemplo podemos citar os arquivos VCFs disponibilizados pelos pesquisadores James Watson e J Craig Venter que possuem 37 megabytes e 39 megabytes de tamanho respectivamente. Um arquivo VCF gerado pelo nosso laboratório contendo o exoma de um único individuo possui em média 10mb de tamanho, isso porque este arquivo contém algumas informação extras sobre as variantes além do genótipo do indivíduo para cada posição encontrada.

É importante notar que enquanto um arquivo FASTQ de um exoma humano com 30x de cobertura tem em média 5GB de tamanho, o arquivo VCF final desse mesmo indivíduo com todas variantes identificadas possui apenas 10MB de tamanho. Um arquivo BAM de um exoma com essa cobertura possui em média 10Gb de espaço em disco.

Para o nosso ao banco de dados, que possui mais de 200 indivíduos armazenados, são necessários 100GB de espaço em disco e apenas 29GB para armazenar todos os arquivos VCFs compactados desses indivíduos, esse tamanho também inclui todas as anotações de centenas de outras fontes de informação que foram incorporadas a esses arquivos VCFs para permitir a análise desses dados. Nosso banco de dados contém as mesmas informações presentes nos arquivos VCFs porém de uma forma mais descompactada, não normalizada e indexada para acelerar a busca e recuperação das informações sobre cada indivíduo utilizando nossa interface web de filtragem de variantes.

Em uma palestra em abril de 2015 com o título ``\textit{Genomics, Big Data, and Medicine Seminar Series}'' o pesquisador George Church afirmou que seriam necessários apenas 9 Petabytes de espaço em disco para armazenar todas as informações genômicas de toda a raça humana. Esse número apesar de inicialmente parecer grande pode ser considerado pequeno quando comparado com a enorme quantidade de informações que empresas como o Google ou alguns institutos de pesquisa como o NIH estão acostumadas a trabalhar.

Link: \url{https://www.youtube.com/watch?v=iVG4EaMrXfI}

\section{Sobre a anotação de variantes}

A grande vantagem do Mendel,MD em relação aos outros softwares disponíveis para análise genômica é que ele é capaz de integrar de maneira rápida e eficiente todas as informações necessárias para realizarmos a identificação de variantes durante a anotação dos dados. Para realizarmos essa anotação nós utilizamos no total 20GB de arquivos contendo todas as informações de referência que são adicionadas ao arquivos VCF de entrada pela nossa análise.

O \textit{framework} desenvolvido por este trabalho para anotação dos arquivos VCFs pode ser considerado um dos pontos mais fortes deste trabalho. Para que esta tarefa fosse executada de maneira eficiente e em tempo hábil foi necessário a utilização de técnicas modernas de programação durante o seu desenvolvimento que permitissem o uso de múltiplos núcleos, a paralelização da execução dos processos, o uso de filas de espera, a utilização de um \textit{cluster} de computadores e a sincronização dos dados entre diferentes máquinas.

Uma das grandes vantagens da nossa análise que a diferencia das outras existentes foi a implementação do uso de múltiplos núcleos de processamento para realizarmos a anotação dos dados. Isso reduziu bastante o tempo necessário para realizarmos a anotação e permitiu que ela fosse repetida diversas vezes sempre que novos dados estivessem disponíveis para download.

Um dos problemas com a anotação de variantes é que tanto os dados quanto os programas utilizados para anotação estão sendo desenvolvidos e atualizados de uma maneira constante. Isso significa que a cada nova versão disponível de arquivos VCFs (Ex. 1000Genomes, OMIM, dbSNP ou dbNFSP) é necessário repetir a anotação dos exomas para melhorar a análise dos dados. A cada nova versão de arquivo VCF disponível para download ou de um programa utilizado o ideal seria repetir a anotação para todos os indivíduos do banco de dados. Isso pode ajudar a resolver alguns casos clínicos que ainda não foram solucionados e permite eliminar cada vez mais variantes falso-positivas, reduzindo ainda mais o número de variantes candidatas do paciente que precisam ser investigadas manualmente.

\section{Sobre a filtragem de variantes}

Em relação a filtragem de variantes nossa estratégia principal nesta etapa da análise foi a de dividir essa tarefa em diversas rotinas pequenas de uma maneira que isso facilitasse a integração de diferentes opções e métodos durante a filtragem dos dados. 

Um dos primeiros filtros implementados foi em relação ao tipo da variante pesquisada, se ela seria homozigótica (Ex. 1/1, 2/2, 3/3) ou heterozigótica (0/1, 0/2, 0/3, 1/2) de acordo com a suspeita sobre o caso clínico que estivesse sendo investigado. Essa opção também foi utilizada para diferenciar entre os modelos de herança dominantes e recessivos para cada doença Mendeliana específica. Outro filtro bastante útil foi a criação de um campo de texto onde o usuário pudesse entrar com listas de SNPs (Ex. rs1234, rs334, rs556) ou de genes (Ex. \textit{PITX2, SUCLA2, BRCA1}) para poder visualizar as variantes presentes apenas nesses locais de interesse. Isso permitiu o uso de listas de genes já conhecidos para a identificação variantes relacionadas a doenças Mendelianas e trouxe a possibilidade de excluir SNPs ou genes e eliminar aqueles que com certeza não estariam associados com a doença. 

Uma das coisas que foi observada na filtragem de variantes foi o alto número de variantes nos resultados em genes associados a receptores olfativos e genes pertencentes a família das mucinas. Esses genes podem ser facilmente excluídos da lista de resultados pelo usuário.

A próxima opção implementada foi a possibilidade de excluir variantes que estivessem presentes nos indivíduos ``normais'' que seriam utilizados como controles durante a filtragem de variantes. Isso foi bastante utilizado para os casos clínicos onde o exomas dos pais dos pacientes afetados também foram sequenciados junto com o paciente. Esse método se mostrou bastante efetivo para eliminar as variantes falso-positivas que poderiam estar associadas com um viés da tecnologia do sequencialmente e que com certeza não estariam associadas com a doença investigada.

Essa análise permite a combinação de diferentes opções de filtros para ajudar a encontrar a melhor variante candidata possível. Todas as opções dessa análise foram adicionadas de maneira sequencial (gradativa) através do desenvolvimento de um formulário web (Ex. sliders, checkbox, input text, select).

Um dos atributos mais importantes da filtragem de variantes foi a inserção da biblioteca de Javacript chamada de Select2. O Django possui um plugin chamado ``Django-Select2'' que permite a criação de campos no formulário que funcionem com a opção de auto-completar dos dados (\textit{auto-complete})). Isso permitiu a busca por indivíduos, grupos de indivíduos, genes,  grupos de genes, e doenças Mendelianas durante a filtragem de variantes, de forma que o usuário pudesse digitar, por exemplo, 'exome\_3\_eds' e através do uso de uma requisição por ajax utilizando uma o formato json como resposta, o sistema faz uma consulta no banco de dados e retorna uma lista com todos os registros encontrados, como 'exome\_3\_eds var annotated' no resultado que então pode ser facilmente incluído na lista de indivíduos afetados ou na lista de controles. Isso facilita muito o preenchimento dos dados no formulário e permite a modificação das opções de maneira muito rápida.

\section{Variantes falso-positivas e falso-negativas}

Um dos principais problemas atuais com o sequenciamento e a análise de exomas é o alto número de variantes falsos-positivas que estão presentes nos resultados finais de cada paciente mas foram causadas por um viés existente na tecnologia utilizada para realizar o sequenciamento dos dados. Um dos grande desafios atuais é encontrar uma maneira efetiva de eliminar essas variantes sem jogar fora variantes que sejam realmente verdadeiras e talvez tenham um baixo escore de qualidade. Existem algumas técnicas que podem ser utilizadas para ajudar a reduzir este número, como por exemplo, a definição de um valor de limiar mínimo para os valores de qualidade e de profundidade de leitura das variantes. Para utilizar esses valores de filtragem é importante saber os valores médios de cada exoma que estiver sendo analisado para isso criamos uma interface que faz todos os cálculos dessas métricas para o usuário.

As variantes falso-negativas seriam aquelas que existem ao longo do exoma do paciente, mas que não foram detectadas através daquele tipo de sequenciamento realizado. Em alguns casos a variante causadora da doença Mendeliana pode não ter sido coberta por aquele kit de captura que foi utilizado. O tamanho da região coberta e o preço dos kits são duas coisas que variam bastante de acordo com cada empresa que produz os reagentes.

Não podemos esquecer ainda existem que muitas variantes que estão associadas a doenças complexas e podem não estar presentes na região exônica. Por causa disso algumas variantes não poderão ser detectadas através deste método. Antes de realizarmos o sequenciamento de um paciente, o ideal seria sempre fazer um exame chamado ``array de SNPs'' para eliminar a possibilidade de que variações grandes como por exemplo as CNVs sejam as reais causadoras da doença do paciente.

\section{Exportação dos dados}

Outra característica, extremamente importante do Mendel,MD é a possibilidade de se exportar todos os dados referentes às variantes e todas as anotações que foram incorporadas aos arquivos VCF de entrada uma vez que todos os dados forem processados pelo nosso \textit{framework} de anotação. O arquivo final possui centenas de informações incorporadas em cada posição do genoma a partir de outras fontes de dados. Uma das vantagens de gerar um arquivo no formato VCF no final é que ele também poderá ser utilizado normalmente por outros que sejam capazes de realizar esse tipo de análise, mesmo após serem anotados pela nossa ferramenta.

Após o usuário realizar a filtragem dos dados utilizando o Mendel,MD ele pode simplesmente exportar os resultados da filtragem em formato CSV com todos os campos presentes no nosso banco de dados em relação a cada variante para que esses dados possam então ser visualizados por exemplo utilizando programas como o Microsoft Excel ou o LibreOffice Calc. Ao exportar esses dados dados, nós incluímos todas as informações presentes no banco de dados sobre cada variante e não apenas aquelas que são mostradas na visualização dos resultados da filtragem de variantes. Isso ajuda a investigar outras informações sobre essas variantes escolhidas e ajuda a trazer mais evidências para a variante que for escolhida e que pode ser a real causadora da doença Mendeliana estudada.

\section{Visualização dos Dados}

Depois de selecionar uma lista de variantes candidatas utilizando o Mendel,MD existem dois programas que podem ser utilizados para visualizarmos as variantes estudantes o IGV desenvolvido pelo Broad Institute e o Genome Browser desenvolvido pelo a empresa GoldenHelix. Ambos podem ser utilizados para visualizarmos a região que possui cada variantes utilizando os arquivos no formato BAM e VCF.

Caso a variante esteja em uma região sem muitos problemas, com alta cobertura e uma boa distribuição de leituras para as sequências \textit{forward} e \textit{reverse} isso ajuda a trazer mais evidência para a análise de que a variante possa de fato existir no paciente.

Uma das bibliotecas utilizados por esse trabalho para fazer a visualização dos dados foi o matplotlib em Python. Utilizando conhecimentos de matemática como por exemplo, o sistema de coordenadas polares, foi possível converter os dados dos pacientes em relação as posições referentes aos cromossomos para uma distância em relação a um ponto fixo e um ângulo em relação a esse ponto. Usando essa metodologia foi possível criar gráficos circulares para visualizar os dados de vários indivíduos de uma maneira equivalente a uma ferramenta chamada de CircosPlot que é bastante utilizada em Bioinformática. A grande diferença entre a nossa abordagem e a do CircosPlot é que nosso programa aceita a entrada de indivíduos no formato VCF e permite uma maior customização da visualização dos dados conforme apresentado na figura \ref{fig:sucla2_view}.

\section{Questões Éticas}

O sequenciamento de exomas também trouxe algumas questões éticas que precisam ser discutidas por este trabalho. Como este método realiza o sequenciamento de todos os genes humanos conhecidos (cerca de 20 mil), muitas vezes isso pode levar ao que chamamos de descobertas ``acidentais'' de variantes, ou seja, aquelas que estejam relacionadas a outros tipos de doenças que são sejam o foco da pesquisa atual que está sendo realizada. Em 2013 a \textit{American College of Medical Genetics and Genomics} (ACMG) \cite{Green2013} publicou uma lista com alguns genes que supostamente deveriam ser obrigatoriamente investigados pelos médicos em busca de mutações causais que pudessem ajudar no tratamento de algumas doenças. Essas mutações são chamadas de ``\textit{clinically actionable}'' e deveriam ser reportadas toda vez em que fosse realizado o sequenciamento do exoma de um paciente. Entre os genes dessa lista podemos citar os genes \textit{BRCA1, BRCA2, TSC2, TP53} entre outros. Apesar da publicação deste documento, muitos médicos ainda preferem não reportar as variantes encontradas nessa lista de genes.

Um estudo recente publicado em 2015 \cite{Middleton2015} com 6944 pessoas de 75 países diferentes mostrou que 98\% dos entrevistados possuem interesse em serem informados sobre doenças graves que possam ser evitadas ou então tratadas. Atualmente existe uma certa pressão para que essas informações descobertas em estudos por exemplo de \textit{clinical-trials} sejam entregues de volta para o participante mesmo quando a variante não esteja relacionada com o estudo que estiver sendo realizado. Esse estudo também mostrou que existe uma grande diferença de opinião entre os profissionais que conduzem as pesquisas e os participantes da pesquisa que foi realizada.

Ainda é preciso lembrar que nem os pacientes nem os profissionais de saúde estão preparados para trabalhar com esse novo tipo de informação e ainda serão necessários alguns anos para que essa técnica se torne mais precisa e confiável. Um dos grandes problemas enfrentados atualmente por esta técnica é a definição de que uma mutação nova que for encontrada seja realmente patogênica e para isso são necessários muitos estudos funcionais que comprovem realmente essa associação entre a variante, o gene e a doença Mendeliana. 

\subsection{Em relação ao número de pacientes}

Um dos problemas com a aplicação deste método para a investigação de pacientes afetados por doenças Mendelianas é a necessidade de um número mínimo de indivíduos afetados. Muitas vezes devido a baixa incidência da doença na população não existem outros pacientes disponíveis para serem sequenciados junto com o paciente e finalmente serem utilizados na filtragem de variantes em busca de mutações em genes que sejam comuns aos indivíduos afetados. Um dos maiores desafios do sequenciamento de exomas é de encontrar um novo gene candidato e a mutação causadora da doença utilizando para isso apenas os dados do exoma de um único indivíduo afetado pela doença. Apesar disso, já existem muitos casos descritos pela literatura onde essa identificação foi possível utilizando apenas um único indivíduo. Lembrando que a mutação pode estar em outra posição de um gene já conhecido como causador da doença.

Mas para que a análise de exomas tenha alto poder estatístico, o ideal seria utilizar sempre o maior número possível de indivíduos que forem afetados pela mesma doença. Isso ajuda a reduzir muito o número de genes e variantes candidatas e facilita bastante a identificação do gene causador da doença Mendeliana. Não podemos esquecer que ao analisarmos vários indivíduos com a mesma doença é possível que sejam  genes diferentes encontrados em cada um dos indivíduos. Por causa disso o ideal é sempre começar a sua análise com um único indivíduo e adicionar novos indivíduos afetados de maneira gradual. Também pode-se utilizar famílias com mais de um indivíduo afetado para facilitar a análise.

\subsection{Validação Experimental}

Apesar da análise de exomas ter se tornado um poderosa ferramenta para investigação clínica de doenças Mendelianas ainda é preciso utilizar diversos outros experimentos para poder comprovar a associação entre a mutação encontrada e a doença que estiver sendo estudada. A análise de exomas pode ser utilizada inicialmente para selecionar bons genes candidatos mas isso não exclui a necessidade de que as mutações precisem ser validadas através de outras técnicas experimentais.

\subsection{Sobre o futuro da análise de exomas e genomas}

Com o barateamento dos custos para se realizar o sequenciamento de exomas, espera-se que essa técnica continue a avançar e trazer soluções cada vez mais rápidas para o diagnóstico clínico de pacientes. O aumento do número de variantes com frequência conhecida em diferentes populações em bancos de dados públicos, promete facilitar a identificação de variantes patogênicas de uma maneira cada vez mais rápida e eficiente. Enquanto não houver softwares que sejam ``amigáveis'' o suficiente para serem utilizados por médicos e pesquisadores, essa técnica não será plenamente adotada na prática clínica.

\section{Custo do Sequenciamento e da Interpretação de Exomas}

Apesar do preço do sequenciamento de exomas estar atualmente em 2015 na faixa de U\$ 445.00 dólares para uma cobertura mínima de 30X, utilizando para isso um sequenciador de DNA modelo Illumina HiSeq, ou então U\$800 dólares para uma cobertura de 100X, esse preço ainda não inclui o custo para se realizar a análise desses dados.

Na verdade não existe um preço definido para se realizar a análise dos dados mas acredita-se que ela custe muito mais do que o valor do sequenciamento dos dados do paciente.

Fonte: \url{https://www.scienceexchange.com/services/whole-exome-seq}

\section{Serviços Online e Softwares Comerciais}

Já existem algumas empresas nos Estados Unidos como por exemplo a DNAnexus (\url{https://www.dnanexus.com/}) que permitem realizar toda a análise dos dados genômicos de maneira totalmente online utilizando apenas um navegador web. Com esse serviço também é possível realizar o \textit{upload} dos dados em formato FASTQ e gerar arquivos BAMs e VCFs utilizando diferentes programas e pipelines de análise de dados diferentes que forem escolhidos pelo usuário.

Um dos softwares mais utilizados para fazer a visualização e a confirmação de variantes pelo nosso laboratório é o Alamut (\url{http://www.interactive-biosoftware.com/alamut-visual/}). Esse programa permite a visualização de arquivos BAMs e VCFs para investigar a região onde a variante foi detectada e uma de suas grandes vantagens é a possibilidade de inserir mutações ao longo do genoma e verificar o seu impacto utilizando diferentes escores de patogenicidade. Isso pode ser utilizado para confirmar os escores calculados por outros programas como por exemplo ANNOVAR, VEP ou dbNFSP.

Também existem alguns outros softwares comerciais como por exemplo Enlis (\url{https://www.enlis.com/} e VarSeq (\url{http://goldenhelix.com/VarSeq/}) que permitem a realização desse tipo de análise de maneira totalmente offline com programas desenvolvidos para Desktop que podem serem executados em qualquer computador com Windows, Linux ou MAC desde que ela tenha os requisitos mínimos para sua execução. 

A desvantagem de utilizar esses programas para Desktop para analisarmos os dados genômicos é que muitos programas ainda precisam de uma máquina com um alto desempenho para realizarem a anotação e análise dos dados e isso precisa ser executados no próprio computador do usuário de uma maneira totalmente nativa. 

Uma das principais vantagens de se utilizar um ferramenta web como o Mendel,MD é que elas são geralmente mais rápidas, por possuírem um bom processador e bastante memória disponível, o que facilita bastante o processamento dos dados e aumenta a velocidade de consultas ao banco de dados.